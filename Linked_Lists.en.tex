%******************************************************************************%
%                                                                              %
%                  sample.en.tex for LaTeX                                     %
%                  Created on : Tue Mar 10 13:27:28 2015                       %
%                  Made by : David "Thor" GIRON <thor@42.fr>                   %
%                                                                              %
%******************************************************************************%

\documentclass{42-en}


%******************************************************************************%
%                                                                              %
%                                    Header                                    %
%                                                                              %
%******************************************************************************%
\begin{document}



                           \title{Linked Lists}
                          \subtitle{The Archnemesis of Arrays}
                       \member{Tim Ngo -}{tingo@student.42.us.org}
                        \member{42 Staff -}{pedago@42.fr}

\summary {
  This is an introduction to \texttt{Linked Lists}.
}

\maketitle

\tableofcontents


%******************************************************************************%
%                                                                              %
%                                  Foreword                                    %
%                                                                              %
%******************************************************************************%
\chapter{Foreword}

    The forewords section of a \texttt{42} subject is usually not
    related in any way to the actual topic of the subject. The idea is
    to share some jokes (often questionable) or something that the
    community might be interested in.\\


    \textbf{TIM YOU WILL NEED TO FIGURE THIS OUT}


    % Spacing in the source code does not influence spacing in the
    % generated pdf. The blank lines aboves and below won't appear.
    % Instead, use \newline (or its shortcut \\) and \newpage to
    % create vertical spacing.





    As a consequence, let's use the forewords section of this sample
    \texttt{42} subject to introduce the contents of this document and
    its goals. In particular, the formating of a trivial
    \texttt{LaTeX} document and the normalized chaptering of our
    subjects. If you read this from the pdf, don't forget to open the
    source file (file \texttt{sample.en.tex}) next to this pdf, in
    order to see behind the scenes, and to understand which command
    generates which result. Otherwise, if you have started with the
    sources, congrats, that's the spirit ! But open the pdf (file
    \texttt{sample.en.pdf}) anyway.\\

    What to do if the file \texttt{sample.en.pdf} is not available ?
    Easy, just compile the source file \texttt{sample.en.tex} using
    the shell command \texttt{make}. Please refer to the documentation
    to set up \texttt{LaTeX} on your system if needed.\\

    If you're not familliar with \texttt{LaTeX}'s syntax, here is a
    fairly exhaustive list of everything you'll need to write your
    subject.\\

%******************************************************************************%
%                                                                              %
%                                 Introduction                                 %
%                                                                              %
%******************************************************************************%
\chapter{Introduction}
    Today's lecture gave us a general overview of linked lists and their
    uses in general coding practice. Linked lists are the foundation for
    more advanced data structures and are commonly used in place of arrays
    for data storage due to their ability to store more information in
    non-contiguous blocks of memory. This makes linked lists a very powerful
    tool for parallel computation.
    \\
    \\
    There are many applications to the different variations of linked lists.
    Many games will store cycling character animations in a circular linked
    list or a circular array. Many operating systems line up processes and 
    jobs in a queue. The desire to simulate genetic mutation and give people
    superpowers by removing and swapping base pairs in DNA strands can best
    be organized by a doubly linked list.
    \\
    \\
    Within the 42 curriculum, the entire graphics branch frequently utilizes
    linked lists to store coordinates of points for an object in a map or
    rendering. In many Unix projects, making a process and job queue for
    operating systems is essential to not making a computer crash.



%******************************************************************************%
%                                                                              %
%                                  Goals                                       %
%                                                                              %
%******************************************************************************%
\chapter{Goals}
    This exercise set contains 7 exercises which include adding and
    removing items, detecting infinite cycles, and sorting a list. I
    hope you read this Tim because I have no idea what I want them to
    accomplish from doing these exercises.
    
    FIX THIS TIM FIX THIS TIM FIX THIS TIM FIX THIS TIM FIX THIS TIM FIX THIS TIM 




%******************************************************************************%
%                                                                              %
%                             General instructions                             %
%                                                                              %
%******************************************************************************%
\chapter{General instructions}
    The following exercises are designed such that written functions are
    to be turned in as standalone functions; they will not be included
    in the class for LinkedLists but will still perform standardized
    operations. All exercises should be written in Python and no external
    function calls should be made except for the first question; only the
    functions supplied in the classes provided to you should be needed.
    \\
    \\
    Both a \texttt{Node} and \texttt{SinglyList} class have been given to
    you for use. The \texttt{SinglyList} class has an in-class iterator defined to
    traverse a list, but you are free to iterate through the list yourself :)
    \\
    \\
    If a function requires the use of the Node or SinglyList class, do not 
    include that code in your file submission. If you use a previous solution
    please make sure to include that function in the file submission.

\newpage
    \section{Node Class}

    	\begin{42pycode}
class Node(object):
	def __init__(self, content):
		if content is None:
			raise ValueError('Node must have content')
		self.c = content
		self.n = None
	
	@property
	def content(self):
		return self.c

	@content.setter
	def content(self, val):
		self.c = val
	
	@property
	def next(self):
		return self.n
	
	@next.setter
	def next(self, val):
		self.n = val
\end{42pycode}

    \section{SinglyList Class}

    	\begin{42pycode}
class SinglyList(object):
	def __init__(self):
		self.h = None

	def __iter__(self):
		current = self.head
		while current:
			yield current
			current = current.next

	@property
	def head(self):
		return self.h

	@head.setter
	def head(self, val):
		self.h = val

	def isEmpty(self):
		return self.head == None

	def add_head(self, node):
		if self.isEmpty():
			self.head = node
		else:
			node.next = self.head
			self.head = node
	\end{42pycode}

\startexercices

\chapter{Exercise\exercicenumber: Print All Nodes in a List}

\extitle{Print All Nodes in a List}
\exnumber{\exercicenumber}
\exfiles{print\_list.py}
\exforbidden{Everything except print() :D}
\exnotes{n/a}

\makeheaderfiles
    Understanding how to traverse a linked list is important to mastering
    its concept. Given the head node of a singly linked list, print out all
    the items in the list.
    \\
    \\
    \textbf{Input Format}

    Complete the function \texttt{print\_list(list\_head)} which takes the head of a list.
    \\
    \\
    \textbf{Output Format}

    Just print out the items in the list in order. If a list is empty do 
    not print anything. No return is needed for the function.

\nextexercice

\chapter{Exercise\exercicenumber: Add an Item to the End of a List}

\extitle{Add an Item to the End of a List}
\exnumber{\exercicenumber}
\exfiles{add\_tail.py}
\exforbidden{Everything :D}
\exnotes{There is a similar function in the SinglyList class to reference :)}

\makeheaderfiles
    You're given the pointer to the head node of a singly linked list and
    the value of a node to add to the list. Create a new node with the given
    value. Insert this node at the tail of the linked list and return the head
    node after the insertion.
    \\
    \\
    \textbf{Input Format}
    
    Complete the function \texttt{add\_tail(list\_head, val)} which takes the head of
    a list and the value to add.
    \\
    \\
    \textbf{Output Format}

    Add the requested value into the back of the list as a node. No return is
    needed for the function.

\nextexercice

\chapter{Exercise\exercicenumber: Remove an Item From a List}

\extitle{Remove an Item From a List}
\exnumber{\exercicenumber}
\exfiles{remove.py}
\exforbidden{Everything :D}
\exnotes{If the list is empty, head will be null. The list will not contain duplicates.}

\makeheaderfiles
    You're given the pointer to the head node of a singly linked list and
    the value of a node to delete from the list. Delete the node with the given
    value if it exists.
    \\
    \\
    \textbf{Input Format}
    
    Complete the function \texttt{remove(list\_head, val)} which takes the head of
    a list and the value to delete. For simplicity, all node values will be numbers.
    \\
    \\
    \textbf{Output Format}
    
    If the node exists, delete that node and link its previous and next references
    together. If the node does not exist, do nothing. No return is needed for the
    function.

\nextexercice

\chapter{Exercise\exercicenumber: Cycle Detection}

\extitle{Cycle Detection}
\exnumber{\exercicenumber}
\exfiles{has\_cycle.py}
\exforbidden{Everything :D}
\exnotes{If the list is empty, head will be null.}

\makeheaderfiles
    In a turn-based multiplayer game, a linked list can be used to cyclically
    repeat player order by having the last player's \texttt{next} reference the
    first player. Check that a given linked list cycles or not.
    \\
    \\
    \textbf{Input Format}
    
    Complete the function \texttt{has\_cycle(list\_head)} which takes the head of
    a list and determines whether the list cycles through the list repeatedly
    or not.
    \\
    \\
    \textbf{Output Format}
    
    If there is a cycle, return True; otherwise, return False.

\nextexercice

\chapter{Exercise\exercicenumber: Merge Two Lists}

\extitle{Merge Two Lists}
\exnumber{\exercicenumber}
\exfiles{merge.py}
\exforbidden{Everything :D}
\exnotes{All trains have at least one car, and no two cars have the same weight.}

\makeheaderfiles
    Two cargo trains arrive in the trainyard and their cargo needs to be
    consolidated into a single train set before it can depart. Both trains
    were organized with the heaviest car in the front of the train
    descending to the lightest car in the back.
    \\
    \\
    \textbf{Input Format}
    
    Complete the function \texttt{merge(train1, train2)} which takes the head
    of two lists train1 and train2 and merges both train sets into a single
    train set. Consider reusing a function you've already written.
    \\
    \\
    \textbf{Output Format}
    
    Return the head of a new train list with all cars sorted by weight from
    heaviest to lightest.

\nextexercice

\chapter{Exercise\exercicenumber: Sort a Linked List}

\extitle{Sort a Linked List}
\exnumber{\exercicenumber}
\exfiles{sort\_asc.py}
\exforbidden{Everything :D}
\exnotes{There is no limitation on which sort to implement for this exercise. You will find some sorts are easier to work into a linked list than others...}

\makeheaderfiles
    Understanding how to organize information in a list is important. Implement
    a sorting algorithm that organizes a linked list with numbers in
    \textbf{ascending} order.
    \\
    \\
    \textbf{Input Format}
    
    Complete the function \texttt{sort\_asc(unsorted\_list)} which takes the head
    of an unsorted list.
    \\
    \\
    \textbf{Output Format}
    
    Sort the list within the function and return nothing.

\chapter{Bonus part}
    Let's try to combine some of the concepts we've gone over and go back to
    a problem we've looked at with a little more difficulty.
FIX THIS TIM FIX THIS TIM FIX THIS TIM FIX THIS TIM FIX THIS TIM FIX THIS TIM 


\nextexercice

\chapter{Exercise\exercicenumber: Trainyard Revisited}

\extitle{Trainyard Revisited}
\exnumber{\exercicenumber}
\exfiles{trainyard.py}
\exforbidden{Everything :D}
\exnotes{All trains have at least one cart, and two carts \textbf{may} have the same weight.}

\makeheaderfiles
    Two more cargo trains arrive in the trainyard and need to be merged before
    departure. These two trains are no longer organized by weight from heaviest
    to lightest prior to the merge. The two lightest cars must also stay behind
    at the trainyard for inspection and leave with the next incoming train.
    \\
    \\
    \textbf{Input Format}
    
    Complete the function \texttt{trainyard(train1, train2)} which takes the head
    of two lists train1 and train2. Consider making more than one function to
    turn in.
    \\
    \\
    \textbf{Output Format}
    
    Return the head of a new train list with all train cars organized from heaviest
    to lightest. Remember to remove the two lightest trains from the set and to
    take into account cars with equal weights.


%******************************************************************************%
%                                                                              %
%                           Turn-in and peer-evaluation                        %
%                                                                              %
%******************************************************************************%
\chapter{Turn-in and peer-evaluation}

    I'm not sure that they do peer evaluation so I'll leave this section in.
FIX THIS TIM FIX THIS TIM FIX THIS TIM FIX THIS TIM FIX THIS TIM FIX THIS TIM 

\end{document}
